\subsection{Classifica��o}\label{classifica��o}
Para classificar os danos como danos de cole�pteros e lagartos foi utilizado o classificador linear LDA(Linear Discriminant Analysis).
O LDA procura a melhor reta de separa��o dos dados de forma que aumente a dist�ncia entre classes e diminua a dist�ncia intra classe.

O LDA calcula os centroides das classes($\mu_{i}$) e o centroide global($\mu$). O c�lculo dos centroides � feito calculando a m�dia sobre os vetores 
de caracter�sticas. Os vetores de caracter�sticas s�o os graus m�ximo e m�dio de conectividade da rede dado um limiar. Tendo os centroides locais, 
os dados s�o normalizados subtraindo-os. Em seguida a matriz de covari�ncia de cada classe � calculada para os dados normalizados.
Depois a matriz de covari�ncia conjunta � calculada utilizando as matrizes de covari�ncia das classes e as respectivas probabilidades a priori.
Tendo esta matriz � calculada a inversa da matriz de covari�ncia ($C^{-1}$). Com isso podemos calcular a fun��o discriminante para cada classe. 
Essa fun��o � calculada pela f�rmula:
\begin{equation}
 f_{i} = \mu_{i}C^{-1}x_{k}^{T}-\frac{1}{2}\mu_{i}C^{-1}\mu_{k}^{T}-ln(p_{i})
\end{equation}\label{eq:funcdisc}

A classe do dano que a ser classificado � aquela que a fun��o discriminante � maximizada dado o vetor de caracter�sticas($x_{k}^{T}$) do dano.